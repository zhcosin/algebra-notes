
\section{行列式}
\label{sec:determinant}

\begin{definition}
  由$1,2,\ldots,n$共$n$组成的一个有序数组称为一个$n$级 \emph{排列}.
\end{definition}

例如,1234和2341都是4级排列,$n$级排列一共有$n!$个.

\begin{definition}
  在一个排列中,如果一对数的大小顺序与先后顺序相反,即前面的数大于后面的数,则称这个数对是一个\emph{逆序},一个排列中的逆序总数,称为该排列的\emph{逆序数}.
\end{definition}

排列$j_1,j_2,\ldots,j_n$的逆序数记为$\tau(j_1,j_2,\ldots,j_n)$.

\begin{definition}
  逆序数为奇数的排列称为\emph{奇排列},逆序数为偶数的排列称为\emph{偶排列}.
\end{definition}

把一个排列中的两个数互换一下位置,称为一次对换操作。

\begin{theorem}
  对换改变排列的奇偶性。
\end{theorem}

\begin{proof}[证明]
  假定原排列是$a_1,a_2,\ldots,a_n$,从中取定一对数$a_i,a_j(i<j)$进行对换,我们对$j-i$作归纳法,如果$j-i=1$,那么对换的是两个相邻的元素,显然逆序增一或者减一,奇偶性自然会改变,假如$j-i\leqslant k$时,对换操作改变排列的奇偶性,那么当$j-i=k+1$时,我们把这个对换操作分成三步来进行,先对换$i$和$j-1$位置上的元素,再对换$j-1$和$j$位置上的元素,最后再对换$i$和$j-1$位置上的元素,这样与直接将$i$和$j$位置上的元素对换得出的排列相同,因而具有相同的逆序数,根据归纳假设,三次对换操作的下标之差都不超过$k$,因而这三次对换将使排列的奇偶性改变三次,等同于改变一次奇偶性,结论得证。
\end{proof}

\begin{inference}
  在$n!$个$n$级排列中,奇偶排列各占一半。
\end{inference}

\begin{proof}[证明]
  通过将前两个元素对换的操作,可以在所有$n$级排列的集合中建立起一一映射,且奇排列和偶排列无交集,故奇排列个数与偶排列个数相同。
\end{proof}

\begin{theorem}
  任何一个$n$级排列,都可以通过一系列对换变成自然排列$1,2,\ldots,n$,且对换次数的奇偶性与原排列的奇偶性相同。
\end{theorem}

\begin{proof}[证明]
  对$n$作归纳法,对$n=2$结论显然成立,假如对级数不超过$n$的排列都有此结论,则对一个$n+1$级排列,如果它的逆序数不是零,则任选一个不在正确位置上的元素,将它与它的正确位置上的元素进行一次对换,在新的排列中除去它,就是一个$n$级的对换,按归纳假设,它可以通过一系列对换变成自然排列,且对换次数与新排列的逆序数相同,再加上第一次对换也改变了一次奇偶性,结论便得到证明。
\end{proof}

%%% Local Variables:
%%% mode: latex
%%% TeX-master: "../algebra-note"
%%% End:
